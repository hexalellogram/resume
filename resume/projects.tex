%-------------------------------------------------------------------------------
%	SECTION TITLE
%-------------------------------------------------------------------------------
\cvsection{Projects}

%-------------------------------------------------------------------------------
%	CONTENT
%-------------------------------------------------------------------------------
\begin{cventries}

%---------------------------------------------------------
  \cvproject
  {ACM Profiles: \normalfont{Social media aggregation and sharing platform \href{https://github.com/acmucsd-projects/acm-profiles-api}{\underline{[GitHub link]}}}} % Project Name + Link
  {Oct. 2020 - Jan. 2021} % Dates
  {
    \begin{cvitems} % Description(s) of experience/contributions/knowledge
      \item {Product Manager: led team of 7 engineers, established deadlines, reviewed all pull requests, and conducted weekly standup meetings}
      \item {DevOps Engineer: built code quality linting and CI/CD pipeline in \textbf{Travis}, deployed \textbf{React} frontend and \textbf{Django} backend to \textbf{Netlify} and \textbf{Heroku}}
      \item {Used \textbf{Docker} and \textbf{docker-compose} alongside PostgreSQL to ease local database testing for the team's developers}
    \end{cvitems}
  }

% %---------------------------------------------------------
%   \cvproject
%     {SafariSideloader: \normalfont{Script to convert WebExtensions into Safari 14.0+ compatible format \href{https://github.com/hexalellogram/SafariSideloader}{[\underline{GitHub link}]}}} % Project Name + Link
%     {Jun. 2020} % Dates
%     {
%       \begin{cvitems} % Description(s) of experience/contributions/knowledge
%         \item {Used \textbf{Python} to download, extract, and convert WebExtensions from the Chrome Web Store for use in Safari 14.0+}
%         \item {Built script as a wrapper around Apple's Xcode conversion tools to automate generation of template applications with Safari extensions}
%       \end{cvitems}
%     }

% %---------------------------------------------------------
%   \cvproject
%     {\href{https://github.com/hexalellogram/FlyerScanner}{FlyerScanner App (SDHacks 2019): \normalfont{AI-powered calendar event generator for Android}}} % Project Name + Link
%     {Oct. 2019} % Dates
%     {
%       \begin{cvitems} % Description(s) of experience/contributions/knowledge
%         \item {Used \textbf{AWS Comprehend ML} to analyze photos of event flyers and extract information such as event titles, descriptions, dates, and times}
%         \item {Employed open source ical4j library to manage events and generate .ics files for event export, in compliance with iCalendar standard}
%         % \item {Integrated with Android camera system libraries for photographic functions}
%         \item {Built application for users to enter data into calendaring applications for increased productivity, without tedious manual entry}
%       \end{cvitems}
%     }

% %---------------------------------------------------------
%   \cvproject
%     {\href{https://github.com/hexalellogram/CATS-OSS}{Cyber Activity Tracking System: \normalfont{Google Form viewing and competition troubleshooting app}}} % Project Name + Link
%     {Jun. 2018 - Jan. 2019} % Dates
%     {
%       \begin{cvitems} % Description(s) of experience/contributions/knowledge
%         \item {Used Google Sheets API, Google Forms, and Google OAuth login technology, with JSON import/export via Google Gson library}
%         \item {Built application for cybersecurity competition administrators to triage and troubleshoot geographically disparate competition issues}
%       \end{cvitems}
%     }
    
%---------------------------------------------------------
\end{cventries}